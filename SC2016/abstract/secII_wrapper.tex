NVIDIA's AmgX library can handle our need in a helpful way.
It provides varying iterative solvers and multigrid on multiple GPUs and is free for non-commercial use.
AmgX, however, has different data structures and usages from PETSc.
Applying AmgX in existing PETSc applications would be an annoying task.
Furthermore, 
since many calculations other than linear solvers may remain on CPUs,
users may want to launch simulations with as many CPU cores as possible;
but numbers of GPUs are typically fewer than that of CPU cores on a heterogenerous platform,
meaning data and linear systems on CPUs need to be merged and scattered
before and after calling solvers on GPUs.
All these indicate a nontrivial modification in those existing CFD codes, 
which is what we tried to avoid.

We wrote a wrapper code to couple AmgX and PETSc. 
This wrapper features a simple usage and capability to exploit all available CPU and GPU resources.
It hides all MPI communications, data conversions, data transfer between CPU and GPU, and memory allocations from users.
In most scenarios,
users can directly replace PETSc's
\lstinline[language=C++, basicstyle=\ttfamily]|KSPSetOperators| and 
\lstinline[language=C++, basicstyle=\ttfamily]|KSPSolve| 
with the wrapper's
\lstinline[language=C++, basicstyle=\ttfamily]|setA(A)| and
\lstinline[language=C++, basicstyle=\ttfamily]|solve(x, rhs)|.
\lstinline[language=C++, basicstyle=\ttfamily]|A|,
\lstinline[language=C++, basicstyle=\ttfamily]|x|, and
\lstinline[language=C++, basicstyle=\ttfamily]|rhs|
remain PETSc matrix and vectors before and after calling the wrapper.
Users need no other coding effort.
Setting up the properties of the solvers (e.g. type of solvers and preconditioners) can be done by an user-specified input script at runtime.

