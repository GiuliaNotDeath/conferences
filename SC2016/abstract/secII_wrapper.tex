NVIDIA's AmgX library provides various iterative solvers and preconditioners
on multiple GPUs, and is free for non-commercial use.
Using AmgX in existing PETSc applications would require considerable
code modifications, due to the different data structures used by the two libraries.
Furthermore, since some calculations other than the linear solvers may remain on CPUs,
users may want to take advantage of all the CPU cores available for these tasks.
If the number of GPUs is different to the number of CPU cores on a 
heterogeneous platform (typically fewer GPUs than cores),
the data and linear systems on CPUs need to be merged and scattered before 
and after calling the solvers on GPUs.
This indicates nontrivial modifications in existing CFD codes, 
which could be costly and a deterrent to adopting GPU hardware.

We developed a wrapper code to couple AmgX and PETSc,  featuring simple
usage and the capability to exploit all available CPU and GPU resources.
It hides from users all MPI communications, data conversions, data transfer between CPU 
and GPU, and memory allocations.
In most scenarios, users can directly replace the PETSc calls to
\lstinline[language=C++, basicstyle=\ttfamily]|KSPSetOperators| and 
\lstinline[language=C++, basicstyle=\ttfamily]|KSPSolve| 
with the wrapper functions
\lstinline[language=C++, basicstyle=\ttfamily]|setA(A)| and
\lstinline[language=C++, basicstyle=\ttfamily]|solve(x, rhs)|.
\lstinline[language=C++, basicstyle=\ttfamily]|A|,
\lstinline[language=C++, basicstyle=\ttfamily]|x|, and
\lstinline[language=C++, basicstyle=\ttfamily]|rhs|
remain PETSc matrix and vectors before and after calling the wrapper.
Users need to make no other coding effort.
Setting up the properties of the solvers (e.g., type of solvers and 
preconditioners) can be done by a user-specified input script at runtime.

