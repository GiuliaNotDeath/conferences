The motivation of this work is a desire to accelerate existing 
PETSc\cite{petsc-web-page}-based CFD (computational fluid dynamics) codes with multi-GPU computing.
Rewriting their code base with CUDA or OpenCL may be onerous
for researchers who focus more on the application of CFD, rather than high-performance computing.
Typically, the most expensive part in CFD codes is solving Poisson-like systems,
for which NVIDIA's AmgX\cite{amgx-web-page} library provides multi-GPU solvers and preconditioners.
We present a wrapper code to bridge AmgX and PETSc so that users can 
run their existing PETSc-based CFD codes on modern heterogeneous platforms, 
with minimal code changes.
Not only does the wrapper code handle converting between AmgX and PETSc
data structures, but it also effectively takes care of the situation where the number
of MPI processes is different than the number of GPUs.

We assessed the overall benefits of using multi-GPU systems in CFD applications
when replacing PETSc with AmgX.
While application speed-up\footnotemark was our top concern, we also studied 
other possible benefits, such as savings in terms of hardware and cost.
To determine such benefits, we used Poisson systems in 2D and 3D as benchmarks
and also a CFD application with our open-source code, PetIBM.

\footnotetext{
    ``Speed-up'' in this work refers to {\it application speed-up}: 
    the acceleration that results from directly replacing PETSc linear solvers 
    with AmgX solvers in an application.
    In other words, 
    these application speed-ups represent the overall reductions of run times 
    that users of the applications can experience.
}
