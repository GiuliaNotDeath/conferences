\documentclass[conference]{IEEEtran}
\usepackage{cite}

% *** GRAPHICS RELATED PACKAGES ***
\ifCLASSINFOpdf
    \usepackage[pdftex]{graphicx}
    \graphicspath{{../}}
    \DeclareGraphicsExtensions{.pdf,.jpeg,.png}
\else
    \usepackage[dvips]{graphicx}
    \graphicspath{{../}}
    \DeclareGraphicsExtensions{.eps}
\fi

% *** MATH PACKAGES ***
\usepackage{amsmath}
\interdisplaylinepenalty=2500

% *** SPECIALIZED LIST PACKAGES ***
\usepackage{algorithmic}

% *** ALIGNMENT PACKAGES ***
\usepackage{array}

% *** SUBFIGURE PACKAGES ***
\ifCLASSOPTIONcompsoc
    \usepackage[caption=false,font=normalsize,labelfont=sf,textfont=sf]{subfig}
\else
    \usepackage[caption=false,font=footnotesize]{subfig}
\fi

% *** FLOAT PACKAGES ***
\usepackage{stfloats}

% *** PDF, URL AND HYPERLINK PACKAGES ***
\usepackage{url}

% correct bad hyphenation here
\hyphenation{op-tical net-works semi-conduc-tor}


\begin{document}

% title
\title{Accelerating PETSc-Based CFD Codes with Multi-GPU Computing}

% author names and affiliations
\author{
    \IEEEauthorblockN{Pi-Yueh Chuang}
    \IEEEauthorblockA{
        Mechanical and Aerospace Engineering\\
        The George Washington University\\
        Washington, DC 20052\\
        Email: pychuang@gwu.edu}
    \and
    \IEEEauthorblockN{Lorena Barba}
    \IEEEauthorblockA{
        Mechanical and Aerospace Engineering\\
        The George Washington University\\
        Washington, DC 20052\\
        Email: labarba@gwu.edu}}

% make the title area
\maketitle


% abstract
\begin{abstract}
    We wrote a wrapper code to couple PETSc and AmgX libraries in order to utilize AmgX's multi-GPU linear solvers in arbitrary PETSc-based applications.
    The wrapper features a simple usage (with normally only two function calls) and capability to exploit all available CPU and GPU resources,
    which is accomplished through converting data structures implicitly between the two libraries,
    hiding MPI communications,
    and merging sub-systems without users' knowing.
    We performed benchmarks to understand application speed-ups after incorporating multi-GPU capability in PETSc applications. 
    Results of benchmarks suggest that, with multi-GPU computing, we can save 
    1) run times and hardware cost (e.g.\ a 6-CPU-core workstation with 1 NVIDIA K40c can compete with a 16-node CPU cluster); and
    2) cloud HPC cost (e.g.\ a benchmark on Amazon EC2 shows a 16x cost saving between CPU and GPU clusters)
    of scientific simulations.
\end{abstract}

% keywords
\begin{IEEEkeywords}
    Multiple GPU, PETSc, AmgX, CFD
\end{IEEEkeywords}


% start the article
\section{Introduction}

Researchers have been developing and running 
PETSc\cite{petsc-web-page, petsc-user-ref, petsc-efficient} 
applications for scientific simulations on CPU clusters for over 20 years.
As GPU computing becomes popular nowadays, 
running scientific simulations on heterogeneous platforms and exploit all CPU and GPU resources may be desired.
PETSc, however, still doesn't have satisfying support of multi-GPU computing,
so moving those mature PETSc applications from CPU platforms to heterogeneous platforms may be tricky.






\subsection{Subsection Heading Here}
Subsection text here.


\subsubsection{Subsubsection Heading Here}
Subsubsection text here.

\section{Conclusion}
The conclusion goes here.




% conference papers do not normally have an appendix


% use section* for acknowledgment
\section*{Acknowledgment}


The authors would like to thank...





% trigger a \newpage just before the given reference
% number - used to balance the columns on the last page
% adjust value as needed - may need to be readjusted if
% the document is modified later
%\IEEEtriggeratref{8}
% The "triggered" command can be changed if desired:
%\IEEEtriggercmd{\enlargethispage{-5in}}

% references section

% can use a bibliography generated by BibTeX as a .bbl file
% BibTeX documentation can be easily obtained at:
% http://mirror.ctan.org/biblio/bibtex/contrib/doc/
% The IEEEtran BibTeX style support page is at:
% http://www.michaelshell.org/tex/ieeetran/bibtex/
%\bibliographystyle{IEEEtran}
% argument is your BibTeX string definitions and bibliography database(s)
%\bibliography{IEEEabrv,../bib/paper}
%
% <OR> manually copy in the resultant .bbl file
% set second argument of \begin to the number of references
% (used to reserve space for the reference number labels box)
\bibliographystyle{IEEEtran}
\bibliography{SC16}

\end{document}


