The starting point of this work is a desire to 
accelerate existing PETSc-based CFD codes with multi-GPU computing.
For those whose research focuses on CFD simulations 
instead of software development or computer science,
rewriting whole code with low level GPU languages may be the least thing they want.
As the most expensive part in CFD codes is typically solving Poisson-like systems,
tackling only this part with libraries providing multi-GPU linear solvers, 
such as NVIDIA's AmgX,
is more acceptable.
The goal here is to write a wrapper to bridge AmgX and PETSc 
so that users can make their existing PETSc-based CFD codes running on modern heterogeneous platforms
and, at the mean time, with only simple modifications in their code.

In additions, we wonder the overall benefits the multi-GPU computing can give us.
While speed-up\footnotemark is always our top concern,
we are also curious about other possible benefits, such as cost saving or hardware saving.
Therefore, we performed a series of benchmarks beside the coding task of the wrapper for this purpose.

\footnotetext{
    Speed-ups in our works refer to {\it application speed-ups}: 
    directly replace PETSc linear solvers with AmgX solvers in an application
    and see what acceleration the application can have. 
    In other words, 
    these application speed-ups represent the overall reductions of run times 
    that users of the applications can experience.
}
